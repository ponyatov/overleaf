\secrel{Demo application: personal task/time planning system}\secdown

\secrel{Distributed architecture with mobile Android client side}

You can download source code and release binary build .apk from

\bigskip
\url{https://github.com/ponyatov/hedge}
\bigskip

System architecture overview see on \autoref{fig:mobile}.
\bigskip

Current client version still does not support graph-driven dynamic GUI, but can
be used for sensor data collection for experimenting with \neo\ scripts and
planning and analysis algorithms.

For client use you must create \neo\ database instance on external server
\note{\href{http://www.graphenedb.com/}{graphenedb.com} can be used with free
account}, and put BOLT connection line into app configuration. Current version
have problems with on-device detached mirror\note{\neo\ for Android required},
so you can use application only in online mode. Online connection to remote
graphdb was done via \neo\ JDBC by using
\href{https://github.com/neo4j-contrib/neo4j-jdbc#minimum-viable-snippet}{this
manual}.

\secrel{What is personal planning}

In everyday life, I constantly find myself failing and forgetting to do a lot of
things. This is not because of the laziness or folly, but because of the many
competing interests that are characteristic of most of people. We have axes in
money thru job, hobby as fallback for achievement self-satisfaction in
professional, activities for health care, obligations to family and friends,
some of us have new interests and thirst for knowledge. So to achieve success we
can rely on the achievements of modern electronics, and make our T-800 CPU from
an Android mobile phone and inference system based on neo4j graph database.

\bigskip
Go to \href{https://www.graphenedb.com/}{graphenedb.com} or other neo4j online
hosting, sign up/login and 
\begin{itemize}[nosep,leftmargin=*]
  \item 
create new Planning database (free hobby sandbox is good for first time):\\
\emph{don't forget to select modern version neo4j 3.2.3},\\
it has some goodies in Cypher syntax and webgui.
  \item 
Create user for remote access via BOLT protocol: \verb|hedge|.
  \item 
Save your password and BOLT access line:\\ 
\verb|bolt://hobby-bahdmkgcjildgbkepggmibpl.dbs.graphenedb.com:24786|
  \item 
Switch to web GUI via Databases/Planning/Overview/Neo4j Browser/Launch
\end{itemize}
\bigskip

\noindent
For doing some scripting you can use samples hosted at\\
\url{https://github.com/ponyatov/article} :

\bigskip
\begin{tabular}{l l}
\verb|d3view.html + lib/|
&
neo4j database visualizer in JavaScript/D3\\&(run make to download offline
libs)
\\
\verb|neo2viz.py| & \py\ client makes GraphViz plots \\
\end{tabular}
\bigskip


First create yourself as center of Universe:
\begin{verbatim}
$ merge
    (:me{ firstname:'Dmitry', secondname:'Ponyatov',
          email:'dponyatov@gmail.com'},
          title:'Ponyatov')
-[:isa]->
    (:class:Person{title:'Person'})
\end{verbatim}

\fig{fig/person1.png}{person1}{You is-a Person}{width=0.5\textwidth}

Not so impressive, so we need some good look for our data schemes. We can do it
using some View\note{I mean Model View Controller template}\ attributes, which
will set view representation.

So, first make query\note{later we will make it triggerable} which will update
all class-looking elements:

\begin{verbatim}
$ match (x)-[r:isa]->(y) set
        r.color='blue',r.fontcolor='blue',
        y:class,y.shape='box',y.fillcolor='lightpink'
    return x,r,y 
\end{verbatim}

And emphasize youself as center element:

\begin{verbatim}
$ match (x:me) set x.fillcolor='green',x.shape='circle'
\end{verbatim}

\fig{fig/person2.png}{person2}{Colorized variant looks
much better}{width=0.5\textwidth}

We use \href{http://graphviz.org/}{GraphViz} so available \emph{view
attributes}\ --- shape and color names see in manual:
\bigskip

\begin{tabular}{l l}
Colors:&\url{http://www.graphviz.org/doc/info/colors.html}\\
Shapes:&\url{http://www.graphviz.org/doc/info/shapes.html}\\
\end{tabular}


\secrel{Android KB}
\secrel{Resource analysis in task management and logistics applications}
\secrel{Geopositioning in logistic applications}

\secup
